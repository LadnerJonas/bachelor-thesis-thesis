% !TeX root = ../main.tex
% Add the above to each chapter to make compiling the PDF easier in some editors.

\chapter{Conclusion}\label{chapter:conclusion}\acresetall
Our benchmarking results indicate that the LocalPagesAndMerge implementation achieves the highest throughput for small partition counts ($\le$32).
However, as the number of partitions increases, its performance deteriorates due to the rising amount of \ac{TLB} misses and the computational overhead of the sort-and-merge phase.
In contrast, the \ac{SMB} implementations (Smb, SmbBatched, SmbLockfree, and SmbLockFreeBatched) initially suffer from contention but improve significantly as the partition count exceeds the physical core count, eventually surpassing LocalPagesAndMerge in throughput.

The Hybrid and Radix implementations perform well, but the Radix approach requires materializing the incoming tuple stream.
The Hybrid implementation mitigates full materialization while achieving comparable throughput.
Our approach, CmpWithProcessingUnits, seeks to reduce the contention through selective write-outs.
However, the associated overhead significantly reduces throughput across most benchmarks.
Finally, the CmpExclusivePartitionRanges and OnDemand approaches fail to scale with increasing thread counts, resulting in performance comparable to or worse than single-threaded execution.
\section{Future Work}
Future research should explore adaptive partitioning schemes and other optimization techniques to reduce the \ac{TLB} misses and memory load observed in the LocalPagesAndMerge approach.
Additionally, decreasing contention in SMB-based implementations could further enhance their scalability.
Similarly, optimizing the CmpWithProcessingUnits approach by developing more efficient selective write-out mechanisms may improve its overall performance.

Furthermore, recent studies suggest that the performance of database operators in microbenchmarks can deviate significantly from their behavior in real-world workloads~\parencite{joins-real-system}.
To ensure the practical applicability of our methods, future work should integrate them into a streaming processing engine and evaluate their performance on systems with a larger number of physical cores.
