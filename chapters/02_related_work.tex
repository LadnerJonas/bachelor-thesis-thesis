% !TeX root = ../main.tex
% Add the above to each chapter to make compiling the PDF easier in some editors.

\chapter{Related Work}\label{chapter:related_work}\acresetall
This chapter presents relevant algorithms and approaches, positioning them within the context of our problem domain.

\section{Partitioned Joins}\label{sec-rw-partitioning-joins}
Partitioned joins are a widely used class of hash-based join algorithms~\parencite{joins-real-system, hash-joins-hardware-tuning, equi-join-comparison}, frequently employed in modern database systems~\parencite{umbra, hyper, hyper-compilation}.

A basic partitioned join algorithm consists of a partitioning and a join execution phase.
In the first phase, the algorithm partitions the build and probe sides using a hash key derived from the join condition.
The subsequent join execution phase uses the property that only tuples within the same partition must be compared, improving locality and reducing unnecessary checks.
The final output consists of the union of each partition’s results.

As the underlying hardware significantly impacts the performance of partitioned join algorithms, recent publications have proposed hardware-conscious implementations~\parencite{hash-joins-hardware-tuning, radix-partitioning-case, main-memory-partitioning, data-partitioning-in-memory-systems}.
In the following chapter (see Chapter \ref{chapter:implementations}), we incorporate hardware-conscious approaches from these papers and tailor them to the requirements of the streaming shuffle operator.

\section{Radix Partitioning}\label{sec-rw-radix-partitioning}
Radix partitioning is a fundamental technique to efficiently partition a large dataset~\parencite{radix-partitioning, radix-partitioning-case, data-partitioning-in-memory-systems, partitioned-parallel-radix-sort}.

Its naive, fixed-size tuple version uses three phases: histogram construction, offset calculation, and a write-out phase~\parencite{radix-partitioning-case}.
The first pass constructs a frequency map, the so-called histogram, that stores the number of tuples for each partition.
Using this histogram, the algorithm computes a prefix-sum array to determine partition offsets.
In the last phase, it allocates a memory block large enough to store all tuples.
The histogram and the offset prefix-sum provide enough information to write each tuple to its final destination in the newly allocated block.

By grouping writes into a contiguous memory region, radix partitioning improves cache locality and reduces write amplification.
Our approach extends this concept by applying histogram-based partitioning dynamically to streaming tuple workloads (see Section \ref{section-histogram-based-partitioning}).

\section{Software Managed Buffers}\label{sec-rw-smb}
Database systems frequently use Software Managed Buffers in the size of cache lines to improve throughput and avoid \ac{TLB} thrashing~\parencite{db-buffer-management, smb-join, smb-partitioning-sort}.
SMBs reduce \ac{TLB} misses by buffering small batches of tuples before writing them out, reducing the number of memory page accesses.
Prior work has shown that using cache-line-sized buffers significantly reduces TLB pressure and improves throughput~\parencite{what-every-programmer-should-know-about-memory}.

In our shuffle operator, all implementations, except the naive OnDemand approach, leverage SMBs to optimize memory access and improve overall throughput.
